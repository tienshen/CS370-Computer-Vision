\section*{Guidelines}
\paragraph{Submission.} Submit a \emph{single pdf} file via
Gradescope that includes your solutions, figures, and code. The latex
source file for the homework is provided in case you want to modify it
to produce your report. However, you are welcome to use other
typesetting software as long as the final output is a pdf.
For readability you may attach the code printouts at the end of the
solutions within the same pdf.
Similarly figures enable easy comparision of various approaches.
Poorly written or formatted reports will make it harder for us to
evaluate it and may lead to a deduction of credit.


\paragraph{Late policy.}
\begin{itemize}
\item You can use 7 late days, with up to 3 late days per assignment.
\item Once you have used all 7 late days, penalty is 25\% for each additional late day.
\item We will use your latest submission for grading and for calculating your late day usage.
\item There is no bonus if you don't use late days at all.
\end{itemize}


\paragraph{Plagiarism.}
We expect the students not to copy, refer to, or look at the solutions
in preparing their answers. We expect students to want to learn and
not google for answers. See the Universities' guidelines on academic
honesty (\url{https://www.umass.edu/honesty}).
Finally, we also ask you to not post the solutions online as the
problem sets might be used in future.


\paragraph{Collaboration.} The homework must be done individually,
except where otherwise noted in the assignments. 'Individually' means
each student must hand in their own answers, and each student must
write their own code in the programming part of the assignment. It is
acceptable, however, for students to collaborate in figuring out
answers and helping each other solve the problems, for example within
a study group.
We will be assuming that you will be taking the responsibility to make
sure you personally understand the solution to any work arising from
such a collaboration.


\paragraph{Python requirements.}
Our code is tested on Python 3.
The Python code depends on external
packages such as \cmd{scipy}, \cmd{numpy}, and \cmd{scikit-image}.
Take a look at the resources posted on the course page to set up the
appropriate programming environment and tutorial on basic concepts.


\paragraph{Using other programming languages.}
While we have made the starter code available in Python, 
feel free to implement the homework from scratch using your favorite
programming language. For example you are welcome to use Matlab, C, Java,
Octave or Julia, with the caveat that we may be able help you with
debugging.






